\documentclass[a4paper,12pt]{extarticle}

% ใช้แพ็คเกจสำหรับจัดการหัวกระดาษและเลขหน้า
\renewcommand{\normalsize}{\fontsize{16}{20}\selectfont}
\usepackage{fancyhdr}
\usepackage{polyglossia}
\setdefaultlanguage{thai}
\PolyglossiaSetup{thai}{indentfirst=true}
\usepackage{fontspec}
\setmainfont{TH Sarabun New}  % หรือฟอนต์ไทยอื่นๆ ที่คุณติดตั้ง
\XeTeXlinebreaklocale "th_TH"
\usepackage{fancyhdr}
\usepackage{amsmath}
\usepackage{hyperref}
\usepackage{enumitem} %กำหนด [enum]
\usepackage{url}
\usepackage{indentfirst} % เยื้อง
\setlength{\parindent}{1.5em} % กำหนดการเยื้องย่อหน้าแรก
\setlength{\parskip}{0pt}  % ปิดการเว้นบรรทัดระหว่างย่อหน้า
\usepackage{titlesec} % ใช้แพ็กเกจสำหรับปรับขนาดหัวข้อ
\titleformat{\section} % ปรับขนาดหัวข้อ \section
  {\normalfont\fontsize{18}{22}\bfseries}{\thesection}{1em}{}

\usepackage{graphicx} % แพ็กเกจสำหรับรูปภาพ
\usepackage[margin=1in]{geometry} % ตั้งค่าขอบกระดาษ
\usepackage[document]{ragged2e} % ใช้สำหรับการจัดเรียงข้อความ
\setlength{\parskip}{0.5em} % กำหนดระยะห่างระหว่างพารากราฟ

% กำหนด cross ref [enum]
\makeatletter
\renewcommand{\ref}[1]{[\ref{#1}]}
\makeatother

% ข้อมูลหัวกระดาษ
\pagestyle{fancy}
\fancyhf{} % ล้างหัวกระดาษและเลขหน้าเดิม
\fancyhead[L]{\textbf{Literature Review}} % ชื่อหัวข้อด้านซ้าย
\fancyhead[R]{\textbf{K. Thangthong \\ Student ID: 63010054}} % รหัสนักศึกษาด้านขวาและวันที่

% ข้อมูลเลขหน้า
\fancyfoot[C]{\thepage} % เลขหน้าตรงกลางด้านล่าง

% Main Content
\begin{document}

% ปรับขนาดฟอนต์ของ title, author และ date แบบ manual
\vspace*{0.5cm}
{\centering
\begin{figure}[h]
    \centering
    \includegraphics[width=3cm]{images/Seal_of_King_Mongkut's_Institute_of_Technology_Ladkrabang.png}
\end{figure}
\fontsize{24pt}{28pt}\selectfont \textbf{Literature Review} \\ \textbf{การทบทวนวรรณกรรมทางการศึกษางานวิจัยทางด้าน การคำนวณประสิทธิภาพสูง (High-Performance Conputing)} \par
\vspace{12pt}
\fontsize{16pt}{24pt}\selectfont \textbf{ผู้จัดทำ} \\ นางสาว กัญญ์ธรรม ถังทอง \\ รหัสนักศึกษา 63010054 \par
\vspace{12pt}
\fontsize{16pt}{24pt}\selectfont \textbf{นำเสนอ} \\ ดร.อนุพงษ์ บรรจงการ \\ ผศ.ดร.ศรัณย์ อินทโกสุม \par
}

% บทคัดย่อ
\vspace{12pt}
\begin{center}
    {\fontsize{18pt}{20pt}\selectfont
    \textbf{บทคัดย่อ} % สำหรับหัวข้อ
    \par
    }
	\begin{justify}

	\end{justify}
\end{center}
\vspace{12pt}

\newpage

% Sect 1 สารบัญ
\renewcommand{\contentsname}{\centering สารบัญ}
\tableofcontents
\newpage % สั่งให้ขึ้นหน้าใหม่

% Sect 2 บทนำ
\section{บทนำ (Introduction)}

\newpage

% Sect 3 หัวข้อ 1
\section{}

\newpage

% Sect 4 หัวข้อ 2
\section{}

\newpage

% Sect 5 หัวข้อ 3
\section{}

\newpage

% Sect อภิปรายผล
\section{การอภิปรายผล (Discussion)}

\newpage

% Sect 7 ข้อสรุปและข้อเสนอแนะ
\section{ข้อสรุปและข้อเสนอแนะ (Conclusion and Recommendations)}

\newpage

% บรรณานุกรม
\begin{center}
\addcontentsline{toc}{section}{เอกสารอ้างอิง}
\section*{เอกสารอ้างอิง}
\end{center}

	\begin{enumerate} [label={[\arabic*]}]
		\item \textbf{Book Title:} Author Name, \textit{Book Title}, Publisher, Year. \label{book:example1}
		\item \textbf{Book Title:} Author Name, \textit{Book Title}, Publisher, Year. \label{book:example2}
	\end{enumerate}

\end{document}
